\documentclass{article}

% \usepackage{microtype}
\usepackage{mathpartir}
% \usepackage{xspace}
% \usepackage{xparse}
% \usepackage{mathtools}
% \usepackage{scalerel}
% \usepackage{suffix}
% \usepackage{enumitem}

% \let\Bbbk\relax % Apparently amssymb defines \Bbbk and so does newtxmath, which acmart requires.
% \usepackage{amsmath,amsthm,amssymb}

\def\code#1{\texttt{#1}}

\begin{document}
  \section{Project 1: Part 1}
  \subsection{$X^{UNSAT}$ Results}
  \subsubsection{Asymm. and Symm.}

  Let us assume that it is possible for an object property $R$ to be constrained by both Asymmetry and Symmetry.
  Assume that $(x, y) \in R$.
  By Symmetry, this means that $(y, x) \in R$.
  By Asymmetry, this means that $(y, x) \not \in R$; however, we just stated that $(y, x) \in R$, so we have a contradiction.

  \subsubsection{Ref. and Asymm.}

  Let us assume that it is possible for an object property $R$ to be constrained by both Reflexivity and Asymmetry.
  By Reflexivity, this means that $(x, x) \in R$ for any $x$.
  So, let us assume that $(x, x) \in R$ for some arbitrary $x$.
  Asymmetry states that if $(y, z) \in R$, then $(y, z) \not \in R$ for any $y$ and $z$.
  Applying this to our assumption that $(x, x) \in R$ gives us the conclusion that $(x, x) \not \in R$, so we have a contradiction.

  \subsubsection{Irref. and Ref.}

  Let us assume that it is possible for an object property $R$ to be constrained by both Irreflexivity and Reflexivity.
  Reflexivity says that $(x, x) \in R$ for any $x$, so assume that $(x, x) \in R$ for some arbitrary $x$.
  Irreflexivity, on the other hand, states that $(y, y) \not \in R$ for all $y$.
  From this, it follows that $(x, x) \not \in R$, but we already stated that $(x, x) \in R$, so we have a contradiction.

  \subsection{$X^{NS}$ Results}
  \subsubsection{Trans. and Funct.}

  % To show that the Transitive and Functional constraints together may lead to undecidability, let us suppose that we have the object properties \code{hasFather} and \code{hasFatherOrSelf}, as well as the class \code{Person}.
  % \code{hasFather} has a domain and range of \code{Person} and is constrained by the Transitivity and Functional constraints.
  % \code{hasFather} relates a person to their biological father--e.g., \code{Mike~hasFather~Jack} if individual \code{Jack} is the biological father of \code{Mike}.

  % Now, $\code{hasFather}~\sqsubseteq~\code{hasFatherOrSelf}$ and \code{hasFatherOrSelf} is the same as \code{hasFather}, except that it also relates every individual in \code{Person} to itself.

  % Now consider the following object property chain:
  % \begin{mathpar}
  %   \code{hasFather} \circ \code{hasFatherOrSelf} \circ \code{hasFather} \sqsubseteq \code{hasFather}
  % \end{mathpar}

  % This object property chain does not exhibit a strict partial order, so it may result in undecidability.  

  \subsubsection{Trans. and iFunct.}

  

  \subsubsection{Trans. and Asymm.}

  To show that the Transitive and Asymmetric constraints together may lead to undecidability, let us suppose that we have the object properties \code{descendantOf} and \code{descendantOfOrSelf}, as well as the class \code{Person}.
  \code{descendantOf} has a domain and range of \code{Person} and is constrained by the Transitive and Asymmetric constraints.
  
  Now, $\code{descendantOf}~\sqsubseteq~\code{descendantOfOrSelf}$ and \code{descendantOfOrSelf} is identical to \code{descendantOf}, except that it also relates every individual in \code{Person} to itself.

  Now consider the following object property chain:
  \begin{mathpar}
    \code{descendantOf} \circ \code{descendantOfOrSelf} \circ \code{descendantOf} \sqsubseteq \code{descendantOf}
  \end{mathpar}

  This object property chain does not exhibit a strict partial order, so it may result in undecidability.

  \subsubsection{Trans. and Irref.}

  The previous example also works for this case, except that \code{descendantOf} is constrained by Transitivity and Irreflexivity.

\end{document}